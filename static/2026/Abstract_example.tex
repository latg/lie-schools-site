\documentclass[12pt, a4paper]{article}
\usepackage[utf8]{inputenc}
\usepackage{amsfonts, amsmath, amssymb}
\usepackage[russian]{babel}
\pagestyle{plain} \oddsidemargin=-0.54cm \textwidth=17cm
\topmargin=-0.54cm \headheight=0cm \textheight=24cm

\begin{document}
\renewcommand{\refname}{\normalsize{\textbf{Список литературы}}}

\begin{center}
{\large\bf Орбиты борелевской
подгруппы и порядок Брюа на инволюциях}\\
{\bf Игнатьев М.В.}\\
{\bf Самарский государственный университет}\\
\texttt{mihail.ignatev@gmail.com}
\end{center}

Доклад основан на работе автора \cite{Ignatev}.

Пусть $G=\mathrm{GL}_n(\mathbb{C})$ --- полная линейная группа, $B\subset G$ --- борелевская подгруппа$\ldots$

\bigskip\textbf{Теорема.} Пусть $\sigma$, $\tau\in S_n^2$. Тогда $\sigma\leqslant^*\tau$ в том и только в том случае, когда $R_{\sigma}^*\leqslant R_{\tau}^*$.

\bigskip
\begin{thebibliography}{XXXX}
\bibitem{Ignatev} Игнатьев М.В. Ортогональные подмножества классических
систем корней и коприсоединённые орбиты унипотентных групп. //
Мат.~заметки, т. \textbf{86}, вып. 1, 2009, с. 65--80, см. также
arXiv: \texttt{math.RT/0904.2841v2} (2009).
\end{thebibliography}




\end{document}
